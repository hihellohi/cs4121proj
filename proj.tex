\documentclass[a4paper]{article}

\usepackage[english]{babel}
\usepackage[UKenglish]{isodate}
\usepackage[parfill]{parskip}
\usepackage{algorithm}
\usepackage{algpseudocode}
\usepackage{float}
\usepackage{graphicx}
\usepackage{booktabs}
\usepackage{csquotes}
%\usepackage[margin=2.5cm]{geometry}
\usepackage{listings}
\usepackage{enumerate}
\usepackage{siunitx}
\usepackage{amssymb}
\usepackage{amsthm}
\usepackage{mathtools}
\usepackage{appendix}
\usepackage[style=ieee,
            natbib=true,
            backend=biber]{biblatex}

\addbibresource{references.bib}
\usepackage[hidelinks]{hyperref}
\nocite{*}

\DeclarePairedDelimiter{\ceil}{\lceil}{\rceil}
\DeclarePairedDelimiter{\floor}{\lfloor}{\rfloor}

\newcommand\ddfrac[2]{\frac{\displaystyle #1}{\displaystyle #2}}

\newcommand{\nth}{\textsuperscript}
\renewcommand*{\arraystretch}{1.3}

\newtheorem*{lemma}{Lemma}
\newtheorem*{theorem}{Theorem}

\title{Modifying Markov Chains to Model Traffic}
\author{Kevin Zihong Ni\\\texttt{z5025098}}

\begin{document}

\maketitle

\begin{abstract}
	In this paper, we will describe the modification and adaptation of a traditional discrete time Markov Chain to efficiently model how 
	traffic jams evolve based on the live traffic conditions of every road in a network.
	We will show how this model performed experimentally with respect to both accuracy and time.
	We also will identify limitations in the model and techniques to mitigate them.
\end{abstract}

\section{Introduction}
Vehichular route planning is an important service in the software industry, being provided by various companies such as Google, Apple and Uber.
Some of these services collect live traffic data through the gps units in the smartphones of drivers.
This data is used to improve route planning, choosing routes to avoid high traffic areas.
However, traffic conditions in a road network changes over time and areas that are currently experiencing high traffic may clear up over the next minutes and vice versa.
This is problematic for route planning over long distances, 
as an algorithm may produce a route that goes around an existing traffic jam which would have cleared by the time the driver had reached it.  
It is apparent how traffic forecasting systems could be of benefit to these routing algorithms.

While it is possible to predict, with relative accuracy, the conditions of roads based on historical data, 
these models break down in abnormal traffic conditions that may be caused by external events such as accidents or roadworks.
The model this paper explores uses a combination of historical and live traffic data in order to make short to medium term forecasts that take into account any 
abnormal conditions. 

\section{Model}
While the behaviour of individual cars on the street are highly unpredictable (without the knowledge of its intended destination), we notice that the collective behaviour 
of all cars in a road network becomes very predictable.
By this, we mean that the traffic conditions of the entire network is highly dependent on the traffic conditions of the network a few moments in the past.
This is an area which lends itself to Markov Chains well, as shown by previous uses of the technique such as modelling demographic shift.
Markov chains model the changes in the states between discrete timesteps. 
Hence, before we define our model, we must first declare how much real world time (in seconds) does one timestep represent. 
Let this timestep be $t$.

\subsection{Topology}
We assume we are given a set of roads $E$ and a set of intersections $V$ together making a directed graph $G = (V, E)$.
Every edge $e \in E$ also has a length $d(e)$ and a speed $s(e)$ associated with it.
For simplicity, we also assume that $G$ has no self loops or parallel edges. These features can be trivially added to the model if necessary.

We also make the reasonable assumption that $G$ is strongly connected.
This can be done as the alternative implies either one of the following statements:
\begin{itemize}
	\item There are more than one connected components in which case we can create a separate model for each component.
	\item There are more than one strongly connected components which implies that there are places in the network which 
		a vehichle can enter but never leave (or vice versa) which is clearly absurd.
\end{itemize}

The states of our Markov chain will simply be the set of edges $E$, since this is where the overwhelming majority of cars will be at any one time.

\subsubsection{Road Based Graph}
Since the vertices of the graph represented by a Markov Matrix will become the states of that respective Markov Chain, it is necessary to 
transform our intersection based graph $G$ into another, road based graph $G' = (E, T)$ as outlined by \cite{volk}. 
$T$ denotes the set of possible "turns" in $G$, i.e. the set of possible transitions from one road to another.
An example of this transformation is illustrated in figures \ref{fig:int} and \ref{fig:roa}.

\begin{figure}[h]
    \centering
    \includegraphics[width=0.4\textwidth]{intersection.PNG}
    \caption{intersection based graph}
    \label{fig:int}
\end{figure}

\begin{figure}[h]
    \centering
    \includegraphics[width=0.7\textwidth]{road.PNG}
    \caption{road based graph}
    \label{fig:roa}
\end{figure}

\subsubsection{Turn Weights}
We also must assign a weight to each turn $t \in T$. 
This weight will be used as the (base) transition probability in the markov chain. 
When calculating these weights, we assume that their destinations has the capacity to accept an infinite number of cars. 
This number will be fixed to accomadate these road capacities later.
First, we will need to determine what fraction of cars in each road $e \in E$ will leave $e$ in each timestep. 
We define this to be
$$
l(e) = \min \left(1, \ddfrac{t \epsilon s(e)}{d(e)} \right)
$$

where $\epsilon$ is a constant that accounts for small losses in efficiency due to turning and unexpected changes in speed.

We now consider the edges that $e$ is connected to. The combined weights of all turns to these edges must total $l(e)$.
The simplest solution to this would be to just assign $l(e)/\mathrm{outdegree}(e)$ to each turn.
However, here we also have an opportunity to incorporate historical data into our model.
Provided we have the empirical data showing the number of different turns made at each likelihood 
(which should be trivial to implement with mobile phone live traffic collection), we can calculate the relative likelihood of a turn out of $e$ to each turn $e'$, $P(e, e')$.
The weight of the turn $(e, e')$ will then be $P(e, e') l(e)$.

Putting this all together, we can construct $T$ with the following algorithm
\begin{algorithmic}[1]
	\Function{FindTurns}{$G$}
		\State $T \gets \varnothing$
		\For{$v \in V$}
			\For{$e \in \mathrm{incoming}(v)$}
				\For{$'e \in \mathrm{outgoing}(v)$}
					\State $T \gets T \cup \{(e, e', P(e, e') l(e))\}$
				\EndFor
			\EndFor
		\EndFor
		\State return $T$
	\EndFunction
\end{algorithmic}

\subsection{State Transitions}
We define the current state of the road network as a row vector $\mathbf{v}$ where $v_i$ is an estimation of the number of cars on road $i$.
Our algorithm will take the original state of the network $\mathbf{v}^{(0)}$ as input.
To step this model forward in time, we will use a similar approach to that of a Markov Chain model --- 
that is, we determine each subsequent state vector $\mathbf{v}^{(k + 1)}$ by multiplying the
preceding vector $\mathbf{v}^{(k)}$ with a transition matrix.

One naive solution would be to simply be use the adjacency matrix of $G'$ for this.
While this may suffice for low density traffic, a number of problems arise as the capacities of individual roads fill up.
\begin{itemize}
	\item In this model, there is no theoretical limit on how many cars roads can accumulate.
		We would ideally want to have a hard limit on each road $e$ proportional to $l(e)$.
	\item Each road has no limit of cars it can "accept" in one timestep, wheras it should be limited to a constant.
	\item The speed of traffic in the model is not at all affected by the density of the traffic when in reality, the speed should slow down in high traffic conditions.
\end{itemize}

We can address all these problems simultaneously by creating our transition matrix $M^{(k)}$ based on $\mathbf{v}^{(k)}$ such that these limits are respected.
From there we can determine 
$$\mathbf{v}^{(k + 1)} = M^{(k)}\mathbf{v}^{(k)}$$

\subsection{Transition Matrix}
First we define $M^B$ to be our adjacency matrix of $G'$. This matrix represents the flow of vehichles in the network if there was no loss of efficiency due to traffic.
We also need to define a vector $\mathbf{c}$ which represents the capacities of each road. For each road $e$,
$$
\mathbf{c}_e = \frac{l(e)}{\mathrm{cargap}}
$$
where cargap is a constant representing the average gap between stationary cars (including the body of the leading car). In our experiments we used cargap = 8.

All what we have described so far need only be calculated once during initialisation. To construct $M^{(k)}$ given $\mathbf{v}^{(k)}$, we first must calculate
$\mathbf{h}^{(k)} = M^B \mathbf{v}^{(k)}$. This vector represents the maximum flow of traffic, given ideal conditions.
we aim to construct $M^{(k)}$ by reducing the weights of $M^B$ based on $\mathbf{h}^{(k)}$ and $\mathbf{v}^{(k)}$. We need to calculate a "limit" vector, $\mathbf{l}^{(k)}$, where
$\mathbf{l}^{(k)}_e$ is the maximum number of cars that can be accepted by road $e$ in the next timestep.
For each road $e$, to impose a limit on the input capacity, we need to keep $\mathbf{l}^{(k)}_e \leq t/g$,
where $g$ is the "time gap" between vehichles in seconds --- that is, at a fixed point on the road, what is the minimum amount of time after the previous car passing until the next car passes (we used $g = 3$ in our experiments).
To impose a limit on the total capacity, we also need to keep $\mathbf{l}^{(k)}_e \leq \mathbf{c}_e - \mathbf{v}^{(k)}_e$.

We take the min of these two values, resulting in

$$\mathbf{l}^{(k)}_e = \min \left(\frac{t}{g}, \mathbf{c}_e - \mathbf{v}^{(k)}_e \right)$$

The ratios that we need to reduce each column of $M^B$ by can be found by taking the elementwise division of $\mathbf{l}^{(k)}$ and $\mathbf{v}^{(k)}$,
and taking the min of each element and 1. To apply these ratios to $M^B$, we construct a diagonal matrix $R^{(k)}$ such that 
$$
R^{(k)}_{e, e} = \min\left(\frac{\mathbf{l}^{(k)}_e}{\mathbf{v}^{(k)}_e}, 1 \right)
$$
Our change matrix can then be defined as
$$
C^{(k)} = M^B R^{(k)}
$$

Lastly, we need to make our matrix stochastic. To do this, we find the difference between the sum of each row and 1
$$
\mathbf{d}^{(k)} = \mathbf{e} - C^{(k)} \mathbf{e}
$$
where $\mathbf{e}$ is a row vector of length $|E|$ filled with ones.
This vector represents the "discrepancy" in each row of $C{(k)}$. We will resolve this discrepancy by adding $\mathbf{r}^{(k)}$, to the diagonal of $C^{(k)}$, representing
the cars in that road that remain in the same road for the duration of the timestep.

Hence, if we define a matrix $D^{(k)}$ such that $ D^{(k)}_{e, e} = \mathbf{d}^{(k)}_e $ then

$$M^{(k)} = C^{(k)} + D^{(k)}$$

\printbibliography
\end{document}
